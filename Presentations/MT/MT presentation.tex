\documentclass{beamer}

\usetheme{Madrid}
\usecolortheme{default}

\title{Materials Testing during CubeSat Demise}
\author{Claudio Vestini, Fizza Naqvi, Hani Moussa, Alex Berresford}
\institute{University of Oxford}
\date{\today}

\begin{document}

\frame{\titlepage}

\begin{frame}
    \frametitle{Outline}
    \tableofcontents
\end{frame}

\section{Introduction - Objective}
\begin{frame}
    \frametitle{First Slide}
    \begin{itemize}
        \item Point 1
        \item Point 2
        \item Point 3
    \end{itemize}
\end{frame}

\section{Secondary objectives}
\begin{frame}
    \frametitle{Secondary objective}
    Monitoring atmospheric composition changes to see how materials remain in the atmosphere after re-entry.
\end{frame}

\section{Mission profile}

\begin{frame}
    \frametitle{Mission Profile}
    \begin{itemize}
        \item Point 1
        \item Point 2
        \item Point 3
    \end{itemize}
\end{frame}

\begin{frame}
    \frametitle{Trajectory outline}
        \begin{itemize}
        \item CubeSat is launched into a circular orbit at an altitude of 400km.
        \item A deorbit burn transitions the orbit from circular to elliptical, with a target perigee of 100km.
        \item The deorbit burn achieves a semi-major axis reduction to meet reentry.
        \item After the deorbit burn, the CubeSat spins up for even heating during atmospheric reentry.
        
    \end{itemize}
\end{frame}

\begin{frame}
    \frametitle{Overview of Calculations}
    \begin{itemize}
        \item De-orbit burn is performed through the use of cold gas thrusters with a thrust of 25mN
        \item \textbf{Orbital Velocity:}
\[
v = \sqrt{\frac{G M}{r}}
\]

        \item \textbf{Semi-Major Axis After Burn:}
        \[
        a = \frac{r_{\text{initial}} + r_{\text{perigee}}}{2}
        \]
        \item \textbf{Delta-V for Deorbit Burn:}
        \[
        \Delta v = v_{\text{initial}} - v_{\text{final}}
        \]
    \end{itemize}
\end{frame}

\begin{frame}
    \frametitle{Deorbit Burn and Spin Stabilization}
    \begin{itemize}
        \item \textbf{Spin Stabilization:}
        \[
        \tau = I \frac{\omega}{t_{\text{spin-up}}}, \quad I = \frac{1}{6} m L^2
        \]
        \item Spin target: 3 RPM (\( 3 \times \frac{2\pi}{60} \, \mathrm{rad/s} \)).
    \end{itemize}
\end{frame}


\begin{frame}
    \frametitle{Numerical Methods in Simulation}
    \begin{itemize}
        \item MATLAB's \textbf{\texttt{ode45}} solver is used to simulate trajectory dynamics.
        \item \texttt{ode45} implements the \textbf{Dormand-Prince adaptive Runge-Kutta method (4th/5th order)}.
        \item Equations of motion:
        \[
        \begin{aligned}
        &\dot{r} = v_r, \quad \dot{\theta} = \frac{v_\theta}{r}, \\
        &\dot{v_r} = -\frac{\mu}{r^2} + \frac{F_{\text{drag}} \cdot v_r}{m \cdot v}, \\
        &\dot{v_\theta} = -\frac{F_{\text{drag}} \cdot v_\theta}{m \cdot v}.
        \end{aligned}
        \]
    \end{itemize}
\end{frame}

\begin{frame}
    \frametitle{Atmospheric Drag and Density}
    \begin{itemize}
        \item \textbf{Drag Force:}
        \[
        F_{\text{drag}} = \frac{1}{2} \, C_d \, \rho \, A \, v^2
        \]
        \item Atmospheric density models:
        \begin{itemize}
            \item Exponential model for \( h < 100 \, \mathrm{km}: \, \rho = \rho_0 \exp(-h / H) \).
            \item Jacchia J71 model for higher altitudes.
        \end{itemize}
    \end{itemize}
\end{frame}



\begin{frame}
    \frametitle{Sensors for Materials Testing}
    \item 1
\end{frame}


\begin{frame}
    \frametitle{Telemetry}
\end{frame}



\section{Components needed}

\begin{frame}
    \frametitle{List of Components}
\end{frame}

\begin{frame}
    \frametitle{Electronic Components Overview}
    \begin{itemize}
        \item Point 1
        \item Point 2
        \item Point 3
    \end{itemize}
\end{frame}

\begin{frame}
    \frametitle{Temperature Sensing}
\end{frame}

\begin{frame}
    \frametitle{OBC and Battery Choice}
\end{frame}

\section{1U vs 8U}
\begin{frame}
    \frametitle{Telemetry}
    \begin{itemize}
        \item Point 1
        \item Point 2
        \item Point 3
    \end{itemize}
\end{frame}

\section{Project Risks}

\section{Outro}

\end{document}